\section{Eine Bemerkung zum Abschlu�}
\textit{``Mache ich mich verd�chtig, wenn ich meine E-Mails verschl�ssel?''}\\

Eine Frage, die h�ufig gestellt wird, wenn es um verschl�sselte E-Mails geht. Bisher gab es darauf folgende Antwort:\\

\textit{``Man sieht es einer E-Mail nicht an, ob sie verschl�sselt ist oder nicht. Wer bef�rchtet, dass jemand die Mail beschn�ffelt und feststellen k�nnte, dass sie verschl�sselt ist, hat einen Grund mehr, kryptografische Verfahren zu nutzen!''}\\

Aktuelle Ereignisse zeigen, dass diese Frage nicht mehr so einfach beantwortet werden kann. Dem promovierten Soziologen Andrej H. wurde vorgeworfen, Mitglied einer terroristischen Vereinigung nach �129a StGB zu sein. Der Haftbefehl gegen ihn wurde unter anderem mit \textbf{konspirativem Verhalten} begr�ndet, da er seine E-Mails verschl�sselte.\\

Am 21.Mai 2008 wurden in �stereich die Wohnungen von Aktivisten der Tierrechtsszene durchsucht und 10 Personen festgenommen. Der Haftbefehl wurde mit Verdunklungsgefahr begr�ndet, da die Betroffenen z.B. �ber verschl�sselte E-Mails kommunizierten.\\

Am 18.10.07 hat der Bundesgerichtshof (BGH) in seinem Urteil \href{http://juris.bundesgerichtshof.de/cgi-bin/rechtsprechung/document.py?Gericht=bgh&Art=pm&Datum=2007&Sort=3&anz=154&pos=0&nr=41487&linked=bes&Blank=1&file=dokument.pdf}{Az.: StB 34/07} den Haftbefehl gegen Andrej H. aufgehoben und eindeutig festgestellt, dass die Verschl�sselung von E-Mails als Tatverdacht NICHT ausreichend ist, entscheidend sei der Inhalt: \\

\textit{``Ohne eine Entschl�sselung der in den Nachrichten verwendeten Tarnbegriffe und ohne Kenntnis dessen, was bei den - teilweise observierten und auch abgeh�rten - Treffen zwischen dem Beschuldigten und L. besprochen wurde, wird hierdurch eine mitgliedschaftliche Einbindung des Beschuldigten in die 'militante gruppe' jedoch nicht hinreichend belegt.''}\\

Au�erdem geben die Richter des 3. Strafsenat des BGH zu bedenken, dass Andrej H. \textit{``ersichtlich um seine �berwachung durch die Ermittlungsbeh�rden wusste''}. Schon allein deshalb konnte er \textit{``ganz allgemein Anlass sehen''}, seine Aktivit�ten zu verheimlichen. Woher Andrej H. von der �berwachung wusste, steht bei \href{http://annalist.noblogs.org/}{http://annalist.noblogs.org}.\\

Trotz dieses Urteils des BGH bleibt f�r uns ein bitterer Nachgeschmack �ber die Arbeit unser Ermittler und einiger Richter. Zumindest die Ermittlungsrichter sind der Argumentation der Staatsanwaltschaft gefolgt und haben dem Haftbefehl erst einmal zugestimmt.