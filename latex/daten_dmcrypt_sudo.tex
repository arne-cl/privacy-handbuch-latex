\ 
\item Das Tool \textbf{sudo} vereinfacht die Nutzung der pam-mount Scripte. Die Nutzung dieses Tools ist in verschiedenen Distributionen sehr unterschiedlich. W�hrend Ubuntu es f�r fast alles nutzt, was mit root-Rechten ausgef�hrt wird, muss man unter SuSE-Linux das Paket \textit{sudo} erst installieren und konfigurieren.\\

In der Datei \textit{/etc/sudoers} wird festgelegt, wer welches Kommando mit root-Rechten ausf�hren darf. Die folgende Zeile erlaubt es allen Nutzern, verschl�sselte Container einzubinden und wieder zu schlie�en.
\begin{verbatim}
 %users  ALL=NOPASSWD: /sbin/mount.crypt,/sbin/umount.crypt
\end{verbatim}

Die Datei ist mit dem Kommando \textit{visudo} als \textit{root} zu editieren. F�r alle, die mit dem Editor vi nicht vertraut sind, ein kurzer Ablauf einer Session:
\begin{enumerate}
 \item Nach dem Start wechselt die Taste <i> in den Editiermodus.
\item Der Editiermodus wird mit der Taste <ESC> wieder verlassen.
\item Anschlie�end wird die Datei mit \textit{:w <ENTER>} gespeichert.
\item Der Editor wird mit \textit{:p <ENTER>} beendet.
\end{enumerate}
\end{itemize}
 