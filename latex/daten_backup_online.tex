\subsection{Online Backups}
Neben dem Backup auf einem externen Datentr�ger kann man auch Online-Speicher nutzen. Als Cloud-Provider kann ich Wuala empfehlen. Bei DataStorageUnit.com, ADrive.com, rsync.net u.v.a.m. gibt es Angebote ab 3,- Euro monatlich. Wer einen eigenen (V)Server gemietet hat, kann seine Backups auch dort ablegen. Mit Ausnahme von Wuala muss man sich um die Verschl�sselung der Daten vor dem Upload selbst k�mmern.\\
 
Ein Online-Backup ist praktisch, wenn man mit Laptop in ein Land wie USA reist.  Bei der Einreise werden m�glicherweise die Daten der Laptops gescannt und auch kopiert. Die EFF.org empfiehlt, vor der Reise die Festplatte zu ``reinigen`` \footnote{  \href{http://www.eff.org/deeplinks/2008/05/protecting-yourself-suspicionless-searches-while-t}{http://www.eff.org/deeplinks/2008/05/protecting-yourself-suspicionless-searches-while-t}}. Man k�nnte ein Online-Backup erstellen und auf dem eigenen Rechner die Daten sicher(!) l�schen, also \textit{shred} bzw. \textit{wipe} nutzen. Bei Bedarf holt man sich die Daten wieder auf den Laptop. Vor der Abreise wird das Online-Backup aktualisiert und lokal wieder alles gel�scht.\\

Mit dem Gesetzentwurf zum Zugriff auf Bestandsdaten der Telekommunikation (BR-Drs. 664/12) vom 24.10.2012 r�umt die Bundesregierung den Geheimdiensten und Strafverfolgern die M�glichkeit ein, ohne richterliche Pr�fung die Zugangsdaten zum Online-Speicher vom Provider zu verlangen. Um die gespeicherten Daten, die meist aus dem Bereich \textit{privater Lebensf�hrung} stammen, angemessen vor dem Verfassungsschutz zu sch�tzen, ist man auf Selbsthilfe und Verschl�sselung angewiesen.\\

An ein Online-Backup werden deshalb folgende Anforderungen gestellt: 
\begin{itemize}
 \item Das Backup muss auf dem eigenen Rechner ver- und entschl�sselt werden, um die Vertraulichkeit zu gew�hrleisten.
 \item Es sollten nur ge�nderte Daten �bertragen werden, um Zeitbedarf und Traffic auf ein ertr�gliches Ma� zu reduzieren.
\end{itemize}

\textit{Wuala} oder \textit{Team-Drive} k�nnen als privacy-freundliche Cloud-Speicher genutzt werden. \textit{duplicity} ist ein kleines Backuptool f�r Linux, dass die Daten lokal ver- und entschl�sselt, bevor sie in einen beliebigen Cloud-Speicher hochgeladen werden.


