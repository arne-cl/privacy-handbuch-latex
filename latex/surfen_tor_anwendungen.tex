\subsection{Anwenungen f�r anonymes Surfen}
Eine kleine Auswahl von Beispielen soll zeigen, was mit einem anonymisiertem Browser m�glich ist und was man beachten sollte, um die Anonymit�t zu wahren. 

\subsubsection*{Anonyme E-Mails ohne Account}
Um einzelne E-Mails unkompliziert und ohne Einrichten eines Kontos zu empfangen, kann man tempor�re Wegwerf-Adressen nutzen. Es reicht ein Rechts-Klick auf das Eingabefeld der E-Mail Adresse. Im Kontextmen� w�hlt man den Punkt \textit{E-Mail Adresse generieren (tempor�r)}. Es wird in einem neuen Browser Tab die Webseite des Anbieters ge�ffnet und die tempor�re E-Mail Adresse in das Formularfeld eingetragen. Nach dem Absenden des Anmeldeformular wechselt man in den neu ge�ffneten Browser Tab und wartet auf die Best�tigungsmail. Welcher Anbieter trempor�rer Mail-Adressen genutzt wird, kann in der Konfiguration vom JonDoFox festgelegt werden.
\begin{itemize}
 \item \href{https://anonbox.net}{https://anonbox.net} des CCC.
 \item \href{http://www.10minutemail.com/}{www.10minutemail.com}
 \item \href{http://www.sofort-mail.de}{http://www.sofort-mail.de}
\item \href{http://www.trash-mail.com}{http://www.trash-mail.com}
\item \href{http://dodgit.com/}{http://dodgit.com}
\end{itemize}

Anonymes Senden von E-Mails kann via Remailer Webinterface erfolgen: 
\begin{itemize}
 \item \href{https://www.awxcnx.de/anon-email.htm}{https://www.awxcnx.de/anon-email.htm}
 \item \href{https://www.cotse.net/cgi-bin/mixmail.cgi}{https://www.cotse.net/cgi-bin/mixmail.cgi}
\end{itemize}

\subsubsection*{Dateien anonym tauschen und verteilen}
Um anonym gr��ere Dateien zu tauschen, empfehlen die Entwickler die Nutzung von \textit{1-Click-Hostern}. Die folgenden Webdienste verlangen keine Benutzerdaten und setzen keine Freigabe von Cookies oder Javascript voraus:
\begin{itemize}
 \item http://www.turboupload.com/
 \item http://www.filefactory.com/
 \item http://www.share-now.net/
 \item http://files.ww.com/
\end{itemize}

Man ruft die Webseite des Dienstes auf und l�dt die zu tauschende oder anonym zu verteilende Datei auf den Server des Dienstes. Ist der Upload abgeschlossen, erh�lt man eine Link, den man an die Tauschpartner sendet oder anonym ver�ffentlicht. Interessierte k�nnen sich die Datei unter dem Link herunter laden.\\

Bei anonym verteilten Dateien sollte man sicherstellen, dass die Datei keine Meta-Inforamtionen enth�lt, die den Absender verraten. Insbesondere Dokumente von Office-Suiten (MS-Office, OpenOffice.org) sind nicht anonym verteilbar. Sie enthalten umfangreiche Angaben �ber den Bearbeiter. Diese Dokumente anonymisiert man zuverl�ssig, indem sie ausgedruckt und wieder eingescannt werden. Die resultierende Bilddateien sind dann ges�ubert.




