\section{MAC-Adressen}
Die MAC-Adresse ist eine weltweit eindeutige Kennung der Netzwerkschnittstelle. Jede Netz�werk�karte und WLAN-Modul hat eine eindeutige Kennung.
\begin{itemize}
 \item In IPv4 Netzen wird diese Kennung nur bis zum Router/Gateway �bertragen. Im eigenen Home-Netz braucht man sich keine Gedanken machen, wenn man vom DSL-Provider eine IPv4 Adresse zugeteilt bekommt. In fremden WLANs (Internetcafe', Flughafen, Hotel) ist davon auszugehen, dass die MAC-Adressen protokolliert werden k�nnen.
 \item In IPv6 Netzen wird die MAC-Adresse Bestandteil der IP-Adresse, wenn die Privacy Extension for IPv6 nicht aktiviert wurde. Damit wird die IP-Adresse zu einem eindeutig personenbezogenen Merkmal.
\end{itemize}
Es gibt also Gr�nde, die MAC-Adressen der Netzwerkschnittstellen regelm��ig oder bei Bedarf vor dem Login in ein fremdes WLAN zu �ndern, um keine �berfl�ssigen Spuren zu hinterlassen.

\subsection*{MAC-Adress �ndern (Linux)}
Unter Linux gibt es das Tool \textit{macchanger}, dass man mit dem bevorzugten Paketmanager der Distribution installieren kann. Um bei Bedarf die MAC-Adresse des WLAN-Modul zu �ndern, ist als root zuerst der NetworkManager zu stoppen, dann die MAC-Adresse zu �ndern und danach der NetworkManager wieder zu starten.
\begin{verbatim}
 > sudo su
 # service network-manager stop
 # macchanger -a wlan0
 # service network-manager start
 # exit
\end{verbatim} 

Man kann die MAC-Adresse auch beim Booten automatisch �ndern lassen. F�r Debian, Ubuntu und Mint werden die Netzwerkschnittstellen durch Init-Scripte initialisiert. Man kann man folgendes Sys-V-Init Script \textit{macchanger-boot} nutzen, um die MAC-Adresse der WLAN-Schnittstelle zu �ndern.
\begin{verbatim}
 #!/bin/bash
 
 ### BEGIN INIT INFO
 # Provides: macchanger
 # Required-Start: networking
 # Required-Stop:
 # Should-Start:
 # Default-Start: S
 # Default-Stop:
 # Short-Description: Change MAC addresse of WLAN interface
 ### END INIT INFO
 
 PATH=/sbin:/bin:/usr/bin
 . /lib/lsb/init-functions
 
 case $1 in
   restart|reload|force-reload|start)
     log_action_begin_msg "Change MAC address of wlan0"
     /usr/bin/macchanger -a wlan0 > /dev/null
     log_action_end_msg 0
     ;;
 
   stop)
     ;;
 esac
 exit 0
\end{verbatim} 

Nach dem Download kopiert man das Script nach \textit{/etc/init.d}, setzt die Rechte auf ausf�hrbar und f�gt es in den Startprozess ein. Im Terminal sind daf�r folgende Befehle auszuf�hren:
\begin{verbatim}
 > sudo su
 # cp macchanger-boot /etc/init.d/
 # chmod +x /etc/init.d/macchanger-boot
 # insserv macchanger-boot
 # exit
\end{verbatim} 

n Suse Linux kann man den Befehl zum �ndern der MAC-Adresse am Ende des Scriptes \textit{/etc/init.d/boot.local} einf�gen. Die Datei kann man mit einem Texteditor bearbeiten und die folgende Zeile einf�gen:
\begin{verbatim}
 macchanger -a wlan0
\end{verbatim} 

M�chte man statt der MAC-Adresse des WLAN-Moduls die MAC-Adresse der Ethernetschittstelle �ndern, dann ist \textit{wlan0} durch \textit{eth0} zu ersetzen.